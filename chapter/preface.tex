\chapter*{Preface}

Nowadays, software is a vital and inevitable part of the modern world.Presently, software is being used in many fields, including entertainment, financial and business or national defence and so forth, to simplify a large amount of hard work, reduce cost as well as increase productivity.

In fact, although software engineering has been improved and modernized continuously with so many new and advanced technologies and techniques, it is not able to completely satisfy the demand for the use of software. The problem above exists because a software system is always expected to be reliable, easy to change and expand, but has to meet all the requirements or rules of the real system which it represents, meanwhile the real system changes so quickly and unpredictably. 

Actually, real systems tend to be much larger, more complex during their performances. Thus, to develop compact, understandable, high-quality software for such systems, appropriate methods and right strategies should be applied. Recently, a terminology has been mentioned as an approach to the mentioned problem, modularity - meaning a massive system will be divided into a number of parts or modules and the whole system can be seen as a combination of multiple modules. 

Since many big and complicated enterprise software systems have been built using Java technology, Java is now said to be the most suitable platform for such systems. The fact is that Java platform provides a dedicated edition for that kind of development, Java Enterprise Edition or Java EE, which contains a set of standard technologies to support software developers. Despite the big success of Java EE, it is still criticized for the lack of modularity as systems become tremendously large and complex.

To solve the problem of modularity for Java software systems, a technology has been introduced, the OSGi technology, which offers a style of developing Java software systems in a modular way and it opens up a new direction to building software applications.
 
With the devoted help and useful advice from Ph.D. Tran Thi Minh Chau at Software Engineering dept., Faculty of Information Technology, University of Engineering and Techonology, this report is written to introduce the OSGi technology and a demonstration application developed with OSGi.
   

